\documentclass[12pt,a4paper]{article}
\usepackage[utf8]{inputenc}
\usepackage[T1]{fontenc}
\usepackage[french]{babel}
\usepackage{geometry}
\usepackage{xcolor}
\usepackage{tcolorbox}
\usepackage{enumitem}
\usepackage{fancyhdr}
\usepackage{titlesec}
\usepackage{fontawesome5}
\usepackage{hyperref}

\geometry{margin=2cm}

% Couleurs
\definecolor{primary}{RGB}{79, 70, 229}
\definecolor{action}{RGB}{34, 197, 94}
\definecolor{speech}{RGB}{59, 130, 246}
\definecolor{note}{RGB}{234, 179, 8}
\definecolor{darkgray}{RGB}{55, 65, 81}

% Styles des boîtes
\newtcolorbox{actionbox}[1][]{
    colback=action!10,
    colframe=action,
    fonttitle=\bfseries,
    title={\faMousePointer\ Action à effectuer},
    #1
}

\newtcolorbox{speechbox}[1][]{
    colback=speech!10,
    colframe=speech,
    fonttitle=\bfseries,
    title={\faMicrophone\ Ce qu'il faut dire},
    #1
}

\newtcolorbox{notebox}[1][]{
    colback=note!10,
    colframe=note,
    fonttitle=\bfseries,
    title={\faLightbulb\ Note importante},
    #1
}

\newtcolorbox{timebox}[1][]{
    colback=darkgray!10,
    colframe=darkgray,
    fonttitle=\bfseries,
    title={\faClock\ Durée estimée},
    #1
}

% En-tête et pied de page
\pagestyle{fancy}
\fancyhf{}
\fancyhead[L]{\textcolor{darkgray}{\small Script Démo Vidéo - EventHub}}
\fancyhead[R]{\textcolor{darkgray}{\small \thepage}}
\fancyfoot[C]{\textcolor{darkgray}{\small INPT - DATA}}
\renewcommand{\headrulewidth}{0.4pt}
\renewcommand{\footrulewidth}{0.4pt}

% Style des titres
\titleformat{\section}{\Large\bfseries\color{primary}}{\thesection}{1em}{}
\titleformat{\subsection}{\large\bfseries\color{darkgray}}{\thesubsection}{1em}{}

\begin{document}

% Page de titre
\begin{titlepage}
    \centering
    \vspace*{2cm}
    
    {\Huge\bfseries\textcolor{primary}{EventHub}\par}
    \vspace{0.5cm}
    {\Large Plateforme de Gestion d'Événements\par}
    \vspace{2cm}
    
    {\LARGE\bfseries Script de Démonstration Vidéo\par}
    \vspace{1cm}
    
    {\large Guide complet étape par étape\\
    pour la présentation de l'application\par}
    
    \vfill
    
    \begin{tabular}{ll}
        \textbf{Module:} & Sécurité Web \\
        \textbf{Filière:} & DATA \\
        \textbf{Année:} & 2025-2026 \\
    \end{tabular}
    
    \vspace{2cm}
    
    {\large Institut National des Postes et Télécommunications\par}
    {\large Rabat, Maroc\par}
    
    \vspace{1cm}
    {\small Janvier 2026}
\end{titlepage}

\newpage
\tableofcontents
\newpage

% ============================================================================
\section{Préparation Avant l'Enregistrement}
% ============================================================================

\begin{timebox}
Temps de préparation : \textbf{10-15 minutes}
\end{timebox}

\subsection{Vérifications Techniques}

\begin{actionbox}
\begin{enumerate}[leftmargin=*]
    \item Lancer MongoDB : \texttt{mongod} ou démarrer le service MongoDB
    \item Ouvrir un terminal dans le dossier \texttt{backend/} et exécuter :
    \begin{verbatim}
    npm run dev
    \end{verbatim}
    \item Ouvrir un autre terminal dans \texttt{frontend/} et exécuter :
    \begin{verbatim}
    npm run dev
    \end{verbatim}
    \item Ouvrir le navigateur sur \texttt{http://localhost:5173}
    \item Vérifier que l'application fonctionne correctement
    \item Peupler la base de données (optionnel) :
    \begin{verbatim}
    cd backend && npm run seed
    \end{verbatim}
\end{enumerate}
\end{actionbox}

\subsection{Configuration de l'Enregistrement}

\begin{notebox}
\begin{itemize}[leftmargin=*]
    \item Utiliser un logiciel comme \textbf{OBS Studio}, \textbf{Camtasia} ou \textbf{Loom}
    \item Résolution recommandée : \textbf{1920x1080} (Full HD)
    \item Fermer toutes les applications inutiles
    \item Désactiver les notifications
    \item Préparer un micro de bonne qualité
    \item Parler clairement et à un rythme modéré
\end{itemize}
\end{notebox}

\newpage
% ============================================================================
\section{Introduction (30 secondes)}
% ============================================================================

\begin{timebox}
Durée : \textbf{30 secondes}
\end{timebox}

\begin{actionbox}
Afficher la page d'accueil de l'application dans le navigateur.
\end{actionbox}

\begin{speechbox}
<<Bonjour, je vous présente aujourd'hui \textbf{EventHub}, une application web de gestion d'événements que j'ai développée dans le cadre du module de Sécurité Web.

Cette application est construite avec une architecture \textbf{MERN Stack}, c'est-à-dire MongoDB pour la base de données, Express.js pour le backend, React pour le frontend, et Node.js pour l'environnement serveur.

Je vais vous montrer les principales fonctionnalités de l'application.>>
\end{speechbox}

\newpage
% ============================================================================
\section{Présentation de la Page d'Accueil (1 minute)}
% ============================================================================

\begin{timebox}
Durée : \textbf{1 minute}
\end{timebox}

\begin{actionbox}
\begin{enumerate}[leftmargin=*]
    \item Rester sur la page d'accueil
    \item Faire défiler lentement la page pour montrer le contenu
    \item Montrer la barre de navigation
    \item Cliquer sur <<Voir les événements>>
\end{enumerate}
\end{actionbox}

\begin{speechbox}
<<Voici la page d'accueil de l'application. On peut voir :

\begin{itemize}
    \item Une \textbf{barre de navigation} en haut avec le logo, les liens vers les différentes pages, et les boutons de connexion et d'inscription.
    \item Une section \textbf{héro} avec un message d'accueil et un bouton pour accéder aux événements.
    \item Une présentation des \textbf{fonctionnalités principales} de la plateforme.
\end{itemize}

L'interface est moderne et responsive, elle s'adapte à toutes les tailles d'écran. Je vais maintenant vous montrer la liste des événements.>>
\end{speechbox}

\newpage
% ============================================================================
\section{Liste des Événements (1 minute 30)}
% ============================================================================

\begin{timebox}
Durée : \textbf{1 minute 30 secondes}
\end{timebox}

\begin{actionbox}
\begin{enumerate}[leftmargin=*]
    \item Afficher la page des événements (\texttt{/events})
    \item Montrer les cartes d'événements
    \item Utiliser la barre de recherche pour chercher un événement
    \item Filtrer par catégorie (ex: <<Conference>>)
    \item Montrer la pagination s'il y a plusieurs pages
    \item Cliquer sur un événement pour voir les détails
\end{enumerate}
\end{actionbox}

\begin{speechbox}
<<Sur cette page, on peut voir la liste de tous les événements publiés sur la plateforme.

Chaque événement est affiché sous forme de \textbf{carte} avec :
\begin{itemize}
    \item L'image de couverture
    \item Le titre et la catégorie
    \item La date et le lieu
    \item Le prix et le nombre de places disponibles
\end{itemize}

On dispose de plusieurs outils pour trouver un événement :
\begin{itemize}
    \item Une \textbf{barre de recherche} pour chercher par mot-clé
    \item Des \textbf{filtres par catégorie} : conférence, workshop, concert, sport, networking
    \item Un \textbf{tri} par date ou par prix
\end{itemize}

Voyons maintenant le détail d'un événement en cliquant dessus.>>
\end{speechbox}

\newpage
% ============================================================================
\section{Détail d'un Événement (1 minute)}
% ============================================================================

\begin{timebox}
Durée : \textbf{1 minute}
\end{timebox}

\begin{actionbox}
\begin{enumerate}[leftmargin=*]
    \item Afficher la page de détail d'un événement
    \item Faire défiler pour montrer toutes les informations
    \item Montrer le bouton <<S'inscrire>> (qui demande de se connecter)
\end{enumerate}
\end{actionbox}

\begin{speechbox}
<<Voici la page de détail d'un événement. On y trouve toutes les informations importantes :

\begin{itemize}
    \item L'\textbf{image de couverture} en grand format
    \item Le \textbf{titre} et la \textbf{catégorie} de l'événement
    \item La \textbf{description complète}
    \item Les \textbf{dates} de début et de fin
    \item Le \textbf{lieu} avec l'adresse complète
    \item Le \textbf{prix} et la \textbf{capacité} maximale
    \item Le nombre de \textbf{places restantes}
    \item Les informations sur l'\textbf{organisateur}
\end{itemize}

Pour s'inscrire à un événement, il faut d'abord être connecté. Je vais donc vous montrer le processus d'inscription et de connexion.>>
\end{speechbox}

\newpage
% ============================================================================
\section{Inscription d'un Utilisateur (1 minute 30)}
% ============================================================================

\begin{timebox}
Durée : \textbf{1 minute 30 secondes}
\end{timebox}

\begin{actionbox}
\begin{enumerate}[leftmargin=*]
    \item Cliquer sur <<S'inscrire>> dans la barre de navigation
    \item Remplir le formulaire avec des données de test :
    \begin{itemize}
        \item Prénom : \texttt{Ahmed}
        \item Nom : \texttt{Benjelloun}
        \item Email : \texttt{ahmed.test@email.com}
        \item Mot de passe : \texttt{Password123!}
        \item Confirmation : \texttt{Password123!}
    \end{itemize}
    \item Cliquer sur <<S'inscrire>>
    \item Montrer la redirection vers le tableau de bord
\end{enumerate}
\end{actionbox}

\begin{speechbox}
<<Je vais maintenant créer un nouveau compte utilisateur.

Le formulaire d'inscription demande :
\begin{itemize}
    \item Le \textbf{prénom} et le \textbf{nom}
    \item Une \textbf{adresse email} valide
    \item Un \textbf{mot de passe} sécurisé d'au moins 8 caractères
\end{itemize}

Côté sécurité, le mot de passe est \textbf{hashé avec bcrypt} avant d'être stocké en base de données. Personne, même l'administrateur, ne peut voir les mots de passe en clair.

Après l'inscription, l'utilisateur est automatiquement connecté et redirigé vers son tableau de bord. Un \textbf{token JWT} est généré pour maintenir la session.>>
\end{speechbox}

\newpage
% ============================================================================
\section{Connexion d'un Utilisateur (45 secondes)}
% ============================================================================

\begin{timebox}
Durée : \textbf{45 secondes}
\end{timebox}

\begin{notebox}
Si vous venez de vous inscrire, vous pouvez vous déconnecter d'abord pour montrer le processus de connexion, ou passer cette étape.
\end{notebox}

\begin{actionbox}
\begin{enumerate}[leftmargin=*]
    \item Se déconnecter (cliquer sur le menu utilisateur puis <<Déconnexion>>)
    \item Cliquer sur <<Se connecter>>
    \item Entrer les identifiants :
    \begin{itemize}
        \item Email : \texttt{ahmed.test@email.com}
        \item Mot de passe : \texttt{Password123!}
    \end{itemize}
    \item Cliquer sur <<Se connecter>>
\end{enumerate}
\end{actionbox}

\begin{speechbox}
<<Pour se connecter, il suffit d'entrer son email et son mot de passe.

Le système utilise une authentification par \textbf{JWT (JSON Web Tokens)} :
\begin{itemize}
    \item Un \textbf{access token} de courte durée (15 minutes) pour les requêtes
    \item Un \textbf{refresh token} de longue durée (7 jours) pour renouveler l'access token
\end{itemize}

Cette approche est plus sécurisée que les sessions traditionnelles et permet une meilleure scalabilité.>>
\end{speechbox}

\newpage
% ============================================================================
\section{Création d'un Événement (2 minutes)}
% ============================================================================

\begin{timebox}
Durée : \textbf{2 minutes}
\end{timebox}

\begin{actionbox}
\begin{enumerate}[leftmargin=*]
    \item Cliquer sur <<Créer un événement>> dans la navigation ou le tableau de bord
    \item Remplir le formulaire :
    \begin{itemize}
        \item Titre : \texttt{Workshop Intelligence Artificielle}
        \item Catégorie : \texttt{Workshop}
        \item Description : \texttt{Un atelier pratique sur les fondamentaux de l'IA et du Machine Learning. Venez découvrir les outils et techniques modernes.}
        \item Date début : (choisir une date future)
        \item Date fin : (quelques heures après)
        \item Adresse : \texttt{INPT, Avenue Allal El Fassi}
        \item Ville : \texttt{Rabat}
        \item Capacité : \texttt{50}
        \item Prix : \texttt{0} (gratuit)
    \end{itemize}
    \item Ajouter une image de couverture (drag and drop ou clic)
    \item Cliquer sur <<Publier l'événement>>
\end{enumerate}
\end{actionbox}

\begin{speechbox}
<<En tant qu'utilisateur connecté, je peux créer mes propres événements.

Le formulaire de création est complet et permet de définir :
\begin{itemize}
    \item Les \textbf{informations de base} : titre, catégorie, description
    \item Les \textbf{dates} de début et de fin avec validation (la fin doit être après le début)
    \item La \textbf{localisation} complète
    \item La \textbf{capacité} et le \textbf{prix}
    \item Une \textbf{image de couverture} avec upload de fichier
\end{itemize}

Toutes les données sont \textbf{validées côté serveur} avec Joi pour garantir leur intégrité. L'upload d'image est sécurisé avec vérification du type MIME et limite de taille à 5 Mo.

L'événement est maintenant créé et visible dans la liste !>>
\end{speechbox}

\newpage
% ============================================================================
\section{Inscription à un Événement (1 minute)}
% ============================================================================

\begin{timebox}
Durée : \textbf{1 minute}
\end{timebox}

\begin{actionbox}
\begin{enumerate}[leftmargin=*]
    \item Aller sur la page des événements
    \item Choisir un événement (différent de celui qu'on a créé)
    \item Cliquer sur <<S'inscrire>>
    \item Montrer la confirmation d'inscription
    \item Aller dans <<Mes inscriptions>> pour voir l'inscription
\end{enumerate}
\end{actionbox}

\begin{speechbox}
<<Je vais maintenant m'inscrire à un événement.

Il suffit de cliquer sur le bouton <<S'inscrire>> sur la page de détail. Le système vérifie automatiquement :
\begin{itemize}
    \item Que l'utilisateur n'est pas déjà inscrit
    \item Qu'il reste des places disponibles
    \item Que l'événement n'est pas annulé ou terminé
\end{itemize}

Une fois inscrit, je peux voir tous mes événements dans la section <<Mes inscriptions>>. Je peux aussi me désinscrire si je change d'avis.

Le compteur de places disponibles est mis à jour en temps réel.>>
\end{speechbox}

\newpage
% ============================================================================
\section{Gestion de Mes Événements (1 minute)}
% ============================================================================

\begin{timebox}
Durée : \textbf{1 minute}
\end{timebox}

\begin{actionbox}
\begin{enumerate}[leftmargin=*]
    \item Cliquer sur <<Mes événements>> dans le menu
    \item Montrer la liste des événements créés
    \item Cliquer sur <<Modifier>> pour un événement
    \item Faire une petite modification (ex: changer la capacité)
    \item Sauvegarder
    \item Montrer le bouton <<Supprimer>> (sans cliquer)
\end{enumerate}
\end{actionbox}

\begin{speechbox}
<<Dans la section <<Mes événements>>, je retrouve tous les événements que j'ai créés en tant qu'organisateur.

Pour chaque événement, je peux :
\begin{itemize}
    \item \textbf{Voir} les détails et le nombre d'inscrits
    \item \textbf{Modifier} les informations
    \item \textbf{Supprimer} l'événement si nécessaire
\end{itemize}

Les modifications sont soumises aux mêmes validations que la création. Seul l'organisateur ou un administrateur peut modifier un événement.>>
\end{speechbox}

\newpage
% ============================================================================
\section{Tableau de Bord et Statistiques (1 minute)}
% ============================================================================

\begin{timebox}
Durée : \textbf{1 minute}
\end{timebox}

\begin{actionbox}
\begin{enumerate}[leftmargin=*]
    \item Cliquer sur <<Tableau de bord>> dans le menu
    \item Montrer les statistiques affichées
    \item Montrer les graphiques s'il y en a
    \item Faire défiler pour voir toutes les informations
\end{enumerate}
\end{actionbox}

\begin{speechbox}
<<Le tableau de bord offre une vue d'ensemble de mon activité sur la plateforme.

On y trouve des \textbf{statistiques personnalisées} :
\begin{itemize}
    \item Le nombre d'événements que j'ai créés
    \item Le nombre total d'inscriptions à mes événements
    \item Mes propres inscriptions à d'autres événements
    \item Les événements à venir
\end{itemize}

Ces données sont calculées en temps réel par l'API backend et permettent de suivre facilement son activité.>>
\end{speechbox}

\newpage
% ============================================================================
\section{Fonctionnalités Administrateur (1 minute 30)}
% ============================================================================

\begin{timebox}
Durée : \textbf{1 minute 30 secondes}
\end{timebox}

\begin{notebox}
Pour cette section, il faut se connecter avec un compte administrateur. Si vous avez utilisé le seed, les identifiants sont :\\
Email : \texttt{admin@eventhub.com}\\
Mot de passe : \texttt{Admin123!}
\end{notebox}

\begin{actionbox}
\begin{enumerate}[leftmargin=*]
    \item Se déconnecter du compte utilisateur actuel
    \item Se connecter avec le compte admin
    \item Montrer le tableau de bord admin avec les statistiques globales
    \item Montrer qu'on peut voir/modifier tous les événements
    \item Montrer les statistiques avancées
\end{enumerate}
\end{actionbox}

\begin{speechbox}
<<L'application dispose d'un système de \textbf{rôles} avec contrôle d'accès RBAC.

En tant qu'administrateur, j'ai accès à des fonctionnalités supplémentaires :
\begin{itemize}
    \item \textbf{Statistiques globales} de la plateforme
    \item \textbf{Gestion de tous les événements}, pas seulement les miens
    \item \textbf{Vue d'ensemble} des utilisateurs et inscriptions
\end{itemize}

Le contrôle d'accès est géré côté serveur avec des middlewares qui vérifient le rôle de l'utilisateur avant chaque action sensible. Impossible de contourner ces restrictions depuis le frontend.>>
\end{speechbox}

\newpage
% ============================================================================
\section{Aspects Sécurité (1 minute)}
% ============================================================================

\begin{timebox}
Durée : \textbf{1 minute}
\end{timebox}

\begin{actionbox}
\begin{enumerate}[leftmargin=*]
    \item Ouvrir les DevTools du navigateur (F12)
    \item Montrer l'onglet Network pour voir les requêtes API
    \item Montrer l'onglet Application > Local Storage pour voir les tokens
    \item Fermer les DevTools
\end{enumerate}
\end{actionbox}

\begin{speechbox}
<<Je voudrais souligner les \textbf{mesures de sécurité} implémentées dans l'application :

\textbf{Authentification :}
\begin{itemize}
    \item Mots de passe hashés avec \textbf{bcrypt} (12 rounds)
    \item Tokens \textbf{JWT} signés avec clé secrète
    \item Système de \textbf{refresh token} pour limiter l'exposition
\end{itemize}

\textbf{Protection des données :}
\begin{itemize}
    \item \textbf{Validation} de toutes les entrées avec Joi
    \item \textbf{Helmet} pour les headers HTTP sécurisés
    \item \textbf{CORS} configuré pour n'accepter que le frontend autorisé
    \item \textbf{Rate limiting} pour prévenir les attaques par force brute
\end{itemize}

\textbf{Contrôle d'accès :}
\begin{itemize}
    \item Middleware d'authentification sur les routes protégées
    \item Vérification des permissions (RBAC)
    \item Validation de la propriété des ressources
\end{itemize}
>>
\end{speechbox}

\newpage
% ============================================================================
\section{Conclusion (30 secondes)}
% ============================================================================

\begin{timebox}
Durée : \textbf{30 secondes}
\end{timebox}

\begin{actionbox}
\begin{enumerate}[leftmargin=*]
    \item Revenir sur la page d'accueil
    \item Rester sur cette page pendant la conclusion
\end{enumerate}
\end{actionbox}

\begin{speechbox}
<<Pour conclure, \textbf{EventHub} est une application web complète de gestion d'événements qui démontre :

\begin{itemize}
    \item Une architecture \textbf{MERN Stack} moderne et scalable
    \item Des bonnes pratiques de \textbf{sécurité web}
    \item Une interface utilisateur \textbf{intuitive et responsive}
    \item Un système d'\textbf{authentification robuste}
    \item Une \textbf{API REST} bien structurée
\end{itemize}

Merci pour votre attention. Je reste disponible pour toute question.>>
\end{speechbox}

\newpage
% ============================================================================
\section{Récapitulatif des Durées}
% ============================================================================

\begin{table}[h]
\centering
\begin{tabular}{|l|c|}
\hline
\textbf{Section} & \textbf{Durée} \\
\hline
Introduction & 30 sec \\
Page d'accueil & 1 min \\
Liste des événements & 1 min 30 \\
Détail d'un événement & 1 min \\
Inscription utilisateur & 1 min 30 \\
Connexion & 45 sec \\
Création d'événement & 2 min \\
Inscription à un événement & 1 min \\
Mes événements & 1 min \\
Tableau de bord & 1 min \\
Fonctionnalités admin & 1 min 30 \\
Aspects sécurité & 1 min \\
Conclusion & 30 sec \\
\hline
\textbf{TOTAL} & \textbf{$\sim$ 14 minutes} \\
\hline
\end{tabular}
\caption{Durées estimées par section}
\end{table}

\begin{notebox}
\textbf{Conseils pour l'enregistrement :}
\begin{itemize}[leftmargin=*]
    \item Faites un ou deux essais avant l'enregistrement final
    \item Parlez lentement et clairement
    \item N'hésitez pas à faire des pauses
    \item Si vous faites une erreur, continuez ou recommencez la section
    \item Vérifiez le son et la qualité vidéo avant de commencer
    \item Gardez ce script à portée de main pendant l'enregistrement
\end{itemize}
\end{notebox}

\end{document}
