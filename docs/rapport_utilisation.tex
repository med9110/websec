\documentclass[12pt,a4paper]{article}
\usepackage[utf8]{inputenc}
\usepackage[T1]{fontenc}
\usepackage[french]{babel}
\usepackage{geometry}
\usepackage{graphicx}
\usepackage{xcolor}
\usepackage{listings}
\usepackage{hyperref}
\usepackage{fancyhdr}
\usepackage{titlesec}
\usepackage{float}
\usepackage{booktabs}
\usepackage{array}
\usepackage{longtable}
\usepackage{tabularx}
\usepackage{enumitem}
\usepackage{tcolorbox}
\usepackage{fontawesome5}

\geometry{margin=2.5cm}

% Couleurs
\definecolor{primary}{RGB}{79, 70, 229}
\definecolor{secondary}{RGB}{99, 102, 241}
\definecolor{success}{RGB}{34, 197, 94}
\definecolor{warning}{RGB}{234, 179, 8}
\definecolor{danger}{RGB}{239, 68, 68}
\definecolor{darkgray}{RGB}{55, 65, 81}
\definecolor{lightgray}{RGB}{243, 244, 246}
\definecolor{codebg}{RGB}{30, 30, 30}
\definecolor{codetext}{RGB}{212, 212, 212}

% Configuration listings pour JSON
\lstdefinelanguage{json}{
    basicstyle=\small\ttfamily\color{codetext},
    backgroundcolor=\color{codebg},
    frame=single,
    rulecolor=\color{darkgray},
    showstringspaces=false,
    breaklines=true,
    literate=
        *{:}{{{\color{white}:}}}{1}
        {,}{{{\color{white},}}}{1}
        {\{}{{{\color{yellow}\{}}}{1}
        {\}}{{{\color{yellow}\}}}}{1}
        {[}{{{\color{yellow}[}}}{1}
        {]}{{{\color{yellow}]}}}{1}
        {"}{{\color{success}"}}1,
    morestring=[b]",
    stringstyle=\color{success},
}

\lstset{
    basicstyle=\small\ttfamily\color{codetext},
    backgroundcolor=\color{codebg},
    frame=single,
    rulecolor=\color{darkgray},
    showstringspaces=false,
    breaklines=true,
    tabsize=2
}

% Style des liens
\hypersetup{
    colorlinks=true,
    linkcolor=primary,
    urlcolor=primary,
    citecolor=primary
}

% En-tête et pied de page
\pagestyle{fancy}
\fancyhf{}
\fancyhead[L]{\textcolor{darkgray}{\small EventHub - Rapport d'Utilisation}}
\fancyhead[R]{\textcolor{darkgray}{\small \thepage}}
\fancyfoot[C]{\textcolor{darkgray}{\small INPT - DATA}}
\renewcommand{\headrulewidth}{0.4pt}
\renewcommand{\footrulewidth}{0.4pt}

% Style des titres
\titleformat{\section}{\Large\bfseries\color{primary}}{\thesection}{1em}{}
\titleformat{\subsection}{\large\bfseries\color{secondary}}{\thesubsection}{1em}{}
\titleformat{\subsubsection}{\normalsize\bfseries\color{darkgray}}{\thesubsubsection}{1em}{}

% Box pour les méthodes HTTP
\newtcbox{\httpget}{on line, colback=success!20, colframe=success, boxrule=0.5pt, arc=2pt, boxsep=2pt, left=2pt, right=2pt, fontupper=\small\bfseries\ttfamily}
\newtcbox{\httppost}{on line, colback=primary!20, colframe=primary, boxrule=0.5pt, arc=2pt, boxsep=2pt, left=2pt, right=2pt, fontupper=\small\bfseries\ttfamily}
\newtcbox{\httpput}{on line, colback=warning!20, colframe=warning, boxrule=0.5pt, arc=2pt, boxsep=2pt, left=2pt, right=2pt, fontupper=\small\bfseries\ttfamily}
\newtcbox{\httpdelete}{on line, colback=danger!20, colframe=danger, boxrule=0.5pt, arc=2pt, boxsep=2pt, left=2pt, right=2pt, fontupper=\small\bfseries\ttfamily}

% Box pour les notes
\newtcolorbox{notebox}[1][]{colback=primary!5, colframe=primary, title=#1, fonttitle=\bfseries}
\newtcolorbox{warningbox}[1][]{colback=warning!10, colframe=warning, title=#1, fonttitle=\bfseries}
\newtcolorbox{infobox}[1][]{colback=secondary!5, colframe=secondary, title=#1, fonttitle=\bfseries}

\begin{document}

% Page de titre
\begin{titlepage}
    \centering
    \vspace*{2cm}
    
    {\Huge\bfseries\textcolor{primary}{EventHub}\par}
    \vspace{0.5cm}
    {\Large Plateforme de Gestion d'Événements\par}
    \vspace{2cm}
    
    {\LARGE\bfseries Rapport d'Utilisation\par}
    \vspace{1cm}
    
    {\large Guide d'Installation\\
    Documentation API\\
    Manuel Utilisateur\par}
    
    \vfill
    
    \begin{tabular}{ll}
        \textbf{Module:} & Sécurité Web \\
        \textbf{Filière:} & DATA \\
        \textbf{Année:} & 2025-2026 \\
    \end{tabular}
    
    \vspace{2cm}
    
    {\large Institut National des Postes et Télécommunications\par}
    {\large Rabat, Maroc\par}
    
    \vspace{1cm}
    {\small Janvier 2026}
\end{titlepage}

\newpage
\tableofcontents
\newpage

% ============================================================================
\section{Introduction}
% ============================================================================

\textbf{EventHub} est une application web complète de gestion d'événements permettant aux utilisateurs de créer, gérer et participer à des événements. L'application est construite avec une architecture moderne MERN stack.

\subsection{Fonctionnalités Principales}

\begin{itemize}[leftmargin=*]
    \item \textbf{Authentification sécurisée}: Inscription, connexion avec JWT
    \item \textbf{Gestion des événements}: CRUD complet avec filtres avancés
    \item \textbf{Système d'inscription}: Inscription/désinscription aux événements
    \item \textbf{Upload de fichiers}: Images de couverture et avatars
    \item \textbf{Tableau de bord}: Statistiques et graphiques
    \item \textbf{Rôles utilisateurs}: User et Admin avec permissions différenciées
\end{itemize}

\newpage
% ============================================================================
\section{Installation et Configuration}
% ============================================================================

\subsection{Prérequis}

\begin{table}[H]
\centering
\begin{tabular}{|l|l|l|}
\hline
\textbf{Logiciel} & \textbf{Version minimale} & \textbf{Téléchargement} \\
\hline
Node.js & 18.0.0 & \url{https://nodejs.org} \\
MongoDB & 6.0 & \url{https://www.mongodb.com} \\
npm & 9.0.0 & Inclus avec Node.js \\
Git & 2.40 & \url{https://git-scm.com} \\
\hline
\end{tabular}
\caption{Prérequis système}
\end{table}

\subsection{Installation du Backend}

\begin{enumerate}[leftmargin=*]
    \item \textbf{Naviguer vers le dossier backend}
    \begin{lstlisting}[language=bash]
cd eventhub/backend
    \end{lstlisting}
    
    \item \textbf{Installer les dépendances}
    \begin{lstlisting}[language=bash]
npm install
    \end{lstlisting}
    
    \item \textbf{Créer le fichier .env}
    \begin{lstlisting}[language=bash]
# Copier le fichier exemple
cp .env.example .env

# Editer avec vos valeurs
nano .env
    \end{lstlisting}
    
    \item \textbf{Contenu du fichier .env}
    \begin{lstlisting}
PORT=5000
NODE_ENV=development
MONGODB_URI=mongodb://localhost:27017/eventhub
JWT_SECRET=votre-cle-secrete-jwt-256-bits
JWT_EXPIRES_IN=15m
JWT_REFRESH_SECRET=votre-cle-refresh-secrete
JWT_REFRESH_EXPIRES_IN=7d
FRONTEND_URL=http://localhost:5173
UPLOAD_PATH=uploads
MAX_FILE_SIZE=5242880
    \end{lstlisting}
    
    \item \textbf{Lancer le serveur de développement}
    \begin{lstlisting}[language=bash]
npm run dev
    \end{lstlisting}
\end{enumerate}

\begin{notebox}[Note]
Le serveur backend sera accessible sur \texttt{http://localhost:5000}
\end{notebox}
\begin{figure}[H]
    \centering
    \includegraphics[width=1\linewidth]{image.png}
\end{figure}
\subsection{Installation du Frontend}

\begin{enumerate}[leftmargin=*]
    \item \textbf{Naviguer vers le dossier frontend}
    \begin{lstlisting}[language=bash]
cd eventhub/frontend
    \end{lstlisting}
    
    \item \textbf{Installer les dépendances}
    \begin{lstlisting}[language=bash]
npm install
    \end{lstlisting}
    
    \item \textbf{Lancer le serveur de développement}
    \begin{lstlisting}[language=bash]
npm run dev
    \end{lstlisting}
\end{enumerate}

\begin{notebox}[Note]
L'application frontend sera accessible sur \texttt{http://localhost:5173}
\end{notebox}
\begin{figure}[H]
    \centering
    \includegraphics[width=1\linewidth]{1.png}
\end{figure}
\subsection{Initialisation de la Base de Données}

Pour peupler la base de données avec des données de test:

\begin{lstlisting}[language=bash]
cd backend
npm run seed
\end{lstlisting}

\begin{infobox}[Comptes de test créés]
\begin{tabular}{ll}
\textbf{Admin:} & admin@eventhub.ma / Admin123! \\
\textbf{User 1:} & user@eventhub.ma / User123! \\
\textbf{User 2:} & fatima@eventhub.ma / Fatima123! \\
\textbf{User 3:} & omar@eventhub.ma / Omar123! \\
\end{tabular}
\end{infobox}

\newpage
% ============================================================================
\section{Guide Utilisateur}
% ============================================================================

\subsection{Inscription et Connexion}

\subsubsection{Page d'Inscription}
\begin{figure}[H]
    \centering
    \includegraphics[width=1\linewidth]{2.png}
\caption{Formulaire d'inscription}
\end{figure}


\textbf{Champs requis:}
\begin{itemize}
    \item Prénom (2-50 caractères)
    \item Nom (2-50 caractères)
    \item Email (format valide)
    \item Mot de passe (min 8 caractères, 1 majuscule, 1 minuscule, 1 chiffre)
    \item Confirmation du mot de passe
\end{itemize}

\subsubsection{Page de Connexion}
\begin{figure}[H]
    \centering
    \includegraphics[width=1\linewidth]{3.png}
\caption{Formulaire de connexion}
\end{figure}

\subsection{Tableau de Bord}
\begin{figure}[H]
    \centering
    \includegraphics[width=1\linewidth]{4.png}
\caption{Tableau de bord utilisateur}
\end{figure}


Le tableau de bord affiche:
\begin{itemize}
    \item Statistiques générales (événements, inscriptions)
    \item Graphique de tendance des inscriptions
    \item Répartition par catégorie
    \item Événements à venir
    \item Top événements populaires
\end{itemize}

\subsection{Gestion des Événements}

\subsubsection{Liste des Événements}
\begin{figure}[H]
    \centering
    \includegraphics[width=1\linewidth]{5.png}
\caption{Liste des événements avec filtres}
\end{figure}


\textbf{Filtres disponibles:}
\begin{itemize}
    \item Recherche textuelle (titre, description, ville)
    \item Catégorie (Conférence, Atelier, Concert, Sport, Networking, Autre)
    \item Ville
    \item Plage de dates
    \item Plage de prix
\end{itemize}

\subsubsection{Création d'un Événement}
\begin{figure}[H]
    \centering
    \includegraphics[width=1\linewidth]{6.png}
\caption{Formulaire de création d'événement}
\end{figure}


\textbf{Informations requises:}
\begin{itemize}
    \item Titre (5-200 caractères)
    \item Description (20-5000 caractères)
    \item Catégorie
    \item Date et heure de début/fin
    \item Localisation (adresse, ville, code postal, pays)
    \item Capacité (1-100 000)
    \item Prix (optionnel, 0 = gratuit)
    \item Tags (optionnel, max 10)
    \item Image de couverture (optionnel)
\end{itemize}

\subsubsection{Détail d'un Événement}
\begin{figure}[H]
    \centering
    \includegraphics[width=1\linewidth]{7.png}
\caption{Page de détail d'un événement}
\end{figure}


\subsection{Mes Événements et Inscriptions}

\subsubsection{Mes Événements Créés}
\begin{figure}[H]
    \centering
    \includegraphics[width=1\linewidth]{8.png}
\caption{Mes événements créés}
\end{figure}


\subsubsection{Mes Inscriptions}
\begin{figure}
    \centering
    \includegraphics[width=1\linewidth]{9.png}
\caption{Mes inscriptions aux événements}
\end{figure}


\newpage
% ============================================================================
\section{Documentation API}
% ============================================================================

\subsection{Informations Générales}

\begin{table}[H]
\centering
\begin{tabular}{|l|l|}
\hline
\textbf{Base URL} & \texttt{http://localhost:5000/api} \\
\textbf{Format} & JSON \\
\textbf{Authentification} & Bearer Token (JWT) \\
\textbf{Rate Limiting} & 100 requêtes / 15 minutes \\
\hline
\end{tabular}
\caption{Configuration API}
\end{table}

\subsection{Format des Réponses}

\subsubsection{Réponse Succès}
\begin{figure}[H]
    \centering
    \includegraphics[width=1\linewidth]{10.png}
\end{figure}

\subsubsection{Réponse Erreur}
\begin{figure}[H]
    \centering
    \includegraphics[width=1\linewidth]{11.png}
\end{figure}

\subsection{Authentification}

\subsubsection{\httppost{POST} /api/auth/register}

Inscription d'un nouvel utilisateur.
\begin{figure}[H]
    \centering
    \includegraphics[width=1\linewidth]{12.png}
\end{figure}

\subsubsection{\httppost{POST} /api/auth/login}

Connexion utilisateur.
\begin{figure}[H]
    \centering
    \includegraphics[width=1\linewidth]{13.png}
\end{figure}

\subsubsection{\httppost{POST} /api/auth/logout}

Déconnexion (authentifié requis).

\textbf{Headers:} \texttt{Authorization: Bearer <accessToken>}

\subsubsection{\httpget{GET} /api/auth/me}

Récupérer le profil utilisateur courant.

\textbf{Headers:} \texttt{Authorization: Bearer <accessToken>}
\begin{figure}[H]
    \centering
    \includegraphics[width=1\linewidth]{14.png}
\end{figure}
\subsection{Événements}

\subsubsection{\httpget{GET} /api/events}

Liste des événements avec pagination et filtres.

\textbf{Paramètres de requête:}
\begin{table}[H]
\centering
\small
\begin{tabular}{|l|l|l|l|}
\hline
\textbf{Param} & \textbf{Type} & \textbf{Défaut} & \textbf{Description} \\
\hline
page & number & 1 & Numéro de page \\
limit & number & 10 & Éléments par page (max 100) \\
sort & string & -createdAt & Tri (ex: -startDate, title) \\
search & string & - & Recherche textuelle \\
category & string & - & Filtrer par catégorie \\
status & string & - & Filtrer par statut (admin) \\
city & string & - & Filtrer par ville \\
startDateFrom & date & - & Date début minimum \\
startDateTo & date & - & Date début maximum \\
priceMin & number & - & Prix minimum \\
priceMax & number & - & Prix maximum \\
\hline
\end{tabular}
\caption{Paramètres de filtrage des événements}
\end{table}

\textbf{Exemple:}
\begin{figure}[H]
    \centering
    \includegraphics[width=1\linewidth]{15.png}
\end{figure}
\subsubsection{\httpget{GET} /api/events/:id}

Détail d'un événement.

\subsubsection{\httppost{POST} /api/events}

Créer un événement (authentifié).

\textbf{Headers:} \texttt{Authorization: Bearer <accessToken>}
\begin{figure}[H]
    \centering
    \includegraphics[width=1\linewidth]{16.png}
\end{figure}

\subsubsection{\httpput{PUT} /api/events/:id}

Modifier un événement (propriétaire ou admin).

\subsubsection{\httpdelete{DELETE} /api/events/:id}

Supprimer un événement (propriétaire ou admin).

\subsubsection{\httppost{POST} /api/events/:id/register}

S'inscrire à un événement (authentifié).

\textbf{Réponse (201 Created):}
\begin{figure}[H]
    \centering
    \includegraphics[width=1\linewidth]{17.png}
\end{figure}

\subsubsection{\httpdelete{DELETE} /api/events/:id/register}

Se désinscrire d'un événement.

\subsubsection{\httpget{GET} /api/events/:id/registrations}

Liste des inscrits (propriétaire ou admin).

\subsubsection{\httppost{POST} /api/events/:id/cover}

Upload image de couverture (multipart/form-data).

\textbf{Form Data:} \texttt{cover: <fichier image>}

\subsection{Utilisateur}

\subsubsection{\httpget{GET} /api/users/me/events}

Mes événements créés.

\subsubsection{\httpget{GET} /api/users/me/registrations}

Mes inscriptions.

\subsubsection{\httppost{POST} /api/users/me/avatar}

Upload avatar (multipart/form-data).

\subsection{Statistiques}

\subsubsection{\httpget{GET} /api/stats/dashboard}

Statistiques générales du dashboard.

\textbf{Réponse:}
\begin{figure}[H]
    \centering
    \includegraphics[width=1\linewidth]{19.png}
\end{figure}

\subsubsection{\httpget{GET} /api/stats/events/category}

Répartition des événements par catégorie.

\subsubsection{\httpget{GET} /api/stats/events/status}

Répartition par statut.

\subsubsection{\httpget{GET} /api/stats/events/trend}

Tendance de création des événements (12 mois).

\subsubsection{\httpget{GET} /api/stats/registrations/trend}

Tendance des inscriptions (12 mois).

\subsubsection{\httpget{GET} /api/stats/events/top}

Top 5 événements par inscriptions.

\subsubsection{\httpget{GET} /api/stats/events/upcoming}

Événements à venir (30 jours).

\subsubsection{\httpget{GET} /api/stats/admin}

Statistiques admin globales (admin uniquement).

\subsection{Fichiers}

\subsubsection{\httpget{GET} /api/files}

Liste des fichiers de l'utilisateur.

\subsubsection{\httpget{GET} /api/files/:id}

Détail d'un fichier.

\subsubsection{\httpget{GET} /api/files/:id/download}

Télécharger un fichier.

\subsubsection{\httpget{GET} /api/files/:id/preview}

Prévisualiser un fichier (inline).

\subsubsection{\httpdelete{DELETE} /api/files/:id}

Supprimer un fichier.

\subsection{Codes d'Erreur}

\begin{table}[H]
\centering
\small
\begin{tabular}{|l|l|l|}
\hline
\textbf{Code} & \textbf{HTTP} & \textbf{Description} \\
\hline
VALIDATION\_ERROR & 400 & Données invalides \\
INVALID\_CREDENTIALS & 401 & Email ou mot de passe incorrect \\
TOKEN\_EXPIRED & 401 & Token JWT expiré \\
TOKEN\_INVALID & 401 & Token JWT invalide \\
TOKEN\_MISSING & 401 & Token manquant \\
FORBIDDEN & 403 & Accès refusé \\
INSUFFICIENT\_PERMISSIONS & 403 & Permissions insuffisantes \\
NOT\_FOUND & 404 & Ressource non trouvée \\
EVENT\_NOT\_FOUND & 404 & Événement non trouvé \\
USER\_NOT\_FOUND & 404 & Utilisateur non trouvé \\
USER\_EXISTS & 409 & Email déjà utilisé \\
ALREADY\_REGISTERED & 409 & Déjà inscrit \\
EVENT\_FULL & 409 & Événement complet \\
FILE\_TOO\_LARGE & 400 & Fichier trop volumineux \\
FILE\_TYPE\_NOT\_ALLOWED & 400 & Type de fichier non autorisé \\
INTERNAL\_ERROR & 500 & Erreur serveur \\
\hline
\end{tabular}
\caption{Codes d'erreur de l'API}
\end{table}

\newpage
% ============================================================================
\section{Annexes}
% ============================================================================

\subsection{Structure du Projet}

\begin{lstlisting}
eventhub/
|-- backend/
|   |-- src/
|   |   |-- config/          # Configuration (DB, erreurs)
|   |   |-- controllers/     # Controleurs HTTP
|   |   |-- middlewares/     # Middlewares (auth, upload)
|   |   |-- models/          # Modeles Mongoose
|   |   |-- routes/          # Definitions des routes
|   |   |-- services/        # Logique metier
|   |   |-- seeds/           # Scripts de seed
|   |   |-- server.js        # Point d'entree
|   |-- uploads/             # Fichiers uploades
|   |-- package.json
|
|-- frontend/
|   |-- src/
|   |   |-- components/      # Composants React
|   |   |-- contexts/        # Contextes (Auth)
|   |   |-- pages/           # Pages de l'application
|   |   |-- services/        # Services API
|   |   |-- utils/           # Utilitaires
|   |   |-- App.jsx          # Composant principal
|   |   |-- main.jsx         # Point d'entree
|   |-- index.html
|   |-- package.json
|   |-- vite.config.js
|   |-- tailwind.config.js
|
|-- docs/                    # Documentation
|-- README.md
\end{lstlisting}

\subsection{Commandes Utiles}

\begin{table}[H]
\centering
\begin{tabular}{|l|l|}
\hline
\textbf{Commande} & \textbf{Description} \\
\hline
\texttt{npm run dev} & Lancer en mode développement \\
\texttt{npm start} & Lancer en production \\
\texttt{npm run build} & Build frontend production \\
\texttt{npm run seed} & Peupler la base de données \\
\texttt{npm test} & Exécuter les tests \\
\hline
\end{tabular}
\caption{Commandes NPM principales}
\end{table}

\subsection{Types MIME Autorisés}

\begin{itemize}
    \item \texttt{image/jpeg}
    \item \texttt{image/png}
    \item \texttt{image/gif}
    \item \texttt{image/webp}
\end{itemize}

Taille maximale: \textbf{5 MB}

\end{document}
