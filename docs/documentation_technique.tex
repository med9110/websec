\documentclass[12pt,a4paper]{article}
\usepackage[utf8]{inputenc}
\usepackage[T1]{fontenc}
\usepackage[french]{babel}
\usepackage{geometry}
\usepackage{graphicx}
\usepackage{tikz}
\usetikzlibrary{positioning, shapes, arrows.meta, fit, calc, shapes.geometric}
\usepackage{xcolor}
\usepackage{amssymb}
\usepackage{listings}
\usepackage{hyperref}
\usepackage{fancyhdr}
\usepackage{tocloft}
\usepackage{titlesec}
\usepackage{float}
\usepackage{booktabs}
\usepackage{array}
\usepackage{longtable}

\geometry{margin=2.5cm}

% Couleurs
\definecolor{primary}{RGB}{79, 70, 229}
\definecolor{secondary}{RGB}{99, 102, 241}
\definecolor{darkgray}{RGB}{55, 65, 81}
\definecolor{lightgray}{RGB}{243, 244, 246}
\definecolor{codebg}{RGB}{249, 250, 251}

% Style des liens
\hypersetup{
    colorlinks=true,
    linkcolor=primary,
    urlcolor=primary,
    citecolor=primary
}

% En-tête et pied de page
\pagestyle{fancy}
\fancyhf{}
\fancyhead[L]{\textcolor{darkgray}{\small EventHub - Documentation Technique}}
\fancyhead[R]{\textcolor{darkgray}{\small \thepage}}
\fancyfoot[C]{\textcolor{darkgray}{\small INPT - DATA}}
\renewcommand{\headrulewidth}{0.4pt}
\renewcommand{\footrulewidth}{0.4pt}

% Style des titres
\titleformat{\section}{\Large\bfseries\color{primary}}{\thesection}{1em}{}
\titleformat{\subsection}{\large\bfseries\color{secondary}}{\thesubsection}{1em}{}
\titleformat{\subsubsection}{\normalsize\bfseries\color{darkgray}}{\thesubsubsection}{1em}{}

\begin{document}

% Page de titre
\begin{titlepage}
    \centering
    \vspace*{2cm}
    
    {\Huge\bfseries\textcolor{primary}{EventHub}\par}
    \vspace{0.5cm}
    {\Large Plateforme de Gestion d'Événements\par}
    \vspace{2cm}
    
    {\LARGE\bfseries Documentation Technique\par}
    \vspace{1cm}
    
    {\large Schéma de Base de Données\\
    Diagramme de Classes\\
    Diagrammes de Séquence\par}
    
    \vfill
    
    \begin{tabular}{ll}
        \textbf{Module:} & Sécurité Web \\
        \textbf{Filière:} & DATA \\
        \textbf{Année:} & 2025-2026 \\
    \end{tabular}
    
    \vspace{2cm}
    
    {\large Institut National des Postes et Télécommunications\par}
    {\large Rabat, Maroc\par}
    
    \vspace{1cm}
    {\small Janvier 2026}
\end{titlepage}

\newpage
\tableofcontents
\newpage

% ============================================================================
\section{Introduction}
% ============================================================================

Ce document présente la documentation technique de l'application \textbf{EventHub}, une plateforme complète de gestion d'événements développée avec une architecture MERN (MongoDB, Express.js, React, Node.js).

\subsection{Architecture Générale}

L'application suit une architecture trois tiers:
\begin{itemize}
    \item \textbf{Frontend}: Application React avec Vite, TailwindCSS
    \item \textbf{Backend}: API REST Node.js avec Express.js
    \item \textbf{Base de données}: MongoDB avec Mongoose ODM
\end{itemize}

\subsection{Technologies Utilisées}

\begin{table}[H]
\centering
\begin{tabular}{|l|l|l|}
\hline
\textbf{Couche} & \textbf{Technologie} & \textbf{Version} \\
\hline
Frontend & React & 18.2.0 \\
Frontend & Vite & 5.0.10 \\
Frontend & TailwindCSS & 3.4.0 \\
Frontend & React Router & 6.21.1 \\
Backend & Node.js & 18+ \\
Backend & Express.js & 4.18.2 \\
Backend & Mongoose & 8.0.3 \\
Base de données & MongoDB & 6.0+ \\
Authentification & JWT & jsonwebtoken 9.0.2 \\
\hline
\end{tabular}
\caption{Stack technologique}
\end{table}

\newpage
% ============================================================================
\section{Schéma de Base de Données}
% ============================================================================

\subsection{Modèle Entité-Relation}

\begin{figure}[H]
\centering
\begin{tikzpicture}[
    entity/.style={rectangle, draw=primary, fill=primary!10, thick, minimum width=2.5cm, minimum height=0.9cm},
    relationship/.style={diamond, draw=primary, fill=primary!20, aspect=2.5, inner sep=1pt, font=\small},
    every edge/.style={draw, thick},
    node distance=2cm
]

% Entités - disposition en carré
\node[entity] (user) {\textbf{User}};
\node[entity, right=6cm of user] (event) {\textbf{Event}};
\node[entity, below=5cm of user] (file) {\textbf{File}};
\node[entity, below=5cm of event] (registration) {\textbf{Registration}};

% Relations - positionnées pour éviter les chevauchements
\node[relationship, right=2.5cm of user] (organizes) {organise};
\node[relationship, below=2cm of event] (registers) {inscription};
\node[relationship, below=2cm of user] (uploads) {upload};
\node[relationship, below=2.5cm of organizes] (covers) {couverture};

% Liens User-Event (organise)
\draw[thick] (user) -- (organizes) node[midway, above] {1};
\draw[thick] (organizes) -- (event) node[midway, above] {N};

% Liens User-Registration et Event-Registration (inscription)
\draw[thick] (user.south east) -- ++(0.5,-0.5) -- (registers.north west) node[pos=0.3, above] {1};
\draw[thick] (event) -- (registers) node[midway, right] {1};
\draw[thick] (registers) -- (registration) node[midway, right] {N};

% Liens User-File (upload)
\draw[thick] (user) -- (uploads) node[midway, left] {1};
\draw[thick] (uploads) -- (file) node[midway, left] {N};

% Liens File-Event (couverture)
\draw[thick] (file) -- (covers) node[midway, below] {1};
\draw[thick] (covers) -- (event) node[midway, right] {0..1};

\end{tikzpicture}
\caption{Diagramme Entité-Relation simplifié}
\end{figure}

\subsection{Collections MongoDB}

\subsubsection{Collection: Users}

\begin{table}[H]
\centering
\begin{tabular}{|l|l|l|l|}
\hline
\textbf{Champ} & \textbf{Type} & \textbf{Contraintes} & \textbf{Description} \\
\hline
\_id & ObjectId & PK, Auto & Identifiant unique \\
email & String & Unique, Required & Adresse email \\
password & String & Required, Min 8 & Mot de passe hashé (bcrypt) \\
firstName & String & Required, Max 50 & Prénom \\
lastName & String & Required, Max 50 & Nom de famille \\
role & String & Enum: user, admin & Rôle utilisateur \\
avatar & ObjectId & Ref: File & Photo de profil \\
refreshToken & String & Select: false & Token de rafraîchissement \\
isActive & Boolean & Default: true & Statut du compte \\
createdAt & Date & Auto & Date de création \\
updatedAt & Date & Auto & Date de modification \\
\hline
\end{tabular}
\caption{Structure de la collection Users}
\end{table}

\subsubsection{Collection: Events}

\begin{table}[H]
\centering
\small
\begin{tabular}{|l|l|l|l|}
\hline
\textbf{Champ} & \textbf{Type} & \textbf{Contraintes} & \textbf{Description} \\
\hline
\_id & ObjectId & PK, Auto & Identifiant unique \\
title & String & Required, Max 200 & Titre de l'événement \\
description & String & Required, Max 5000 & Description détaillée \\
category & String & Enum (6 valeurs) & Catégorie \\
status & String & Enum (4 valeurs) & État de publication \\
startDate & Date & Required & Date de début \\
endDate & Date & Required, > startDate & Date de fin \\
location.address & String & Required, Max 300 & Adresse \\
location.city & String & Required, Max 100 & Ville \\
location.postalCode & String & Max 20 & Code postal \\
location.country & String & Default: France & Pays \\
capacity & Number & Required, 1-100000 & Capacité maximale \\
price & Number & Min: 0, Default: 0 & Prix en MAD \\
coverImage & ObjectId & Ref: File & Image de couverture \\
organizer & ObjectId & Ref: User, Required & Organisateur \\
registrationCount & Number & Default: 0 & Nombre d'inscrits \\
tags & [String] & Max 10 items & Tags/mots-clés \\
\hline
\end{tabular}
\caption{Structure de la collection Events}
\end{table}

\textbf{Valeurs Enum pour category:} conference, workshop, concert, sport, networking, other

\textbf{Valeurs Enum pour status:} draft, published, cancelled, completed

\subsubsection{Collection: Registrations}

\begin{table}[H]
\centering
\begin{tabular}{|l|l|l|l|}
\hline
\textbf{Champ} & \textbf{Type} & \textbf{Contraintes} & \textbf{Description} \\
\hline
\_id & ObjectId & PK, Auto & Identifiant unique \\
user & ObjectId & Ref: User, Required & Utilisateur inscrit \\
event & ObjectId & Ref: Event, Required & Événement concerné \\
status & String & Enum (3 valeurs) & État de l'inscription \\
registeredAt & Date & Default: now & Date d'inscription \\
createdAt & Date & Auto & Date de création \\
updatedAt & Date & Auto & Date de modification \\
\hline
\end{tabular}
\caption{Structure de la collection Registrations}
\end{table}

\textbf{Index unique:} (user, event) - Empêche les inscriptions multiples

\subsubsection{Collection: Files}

\begin{table}[H]
\centering
\begin{tabular}{|l|l|l|l|}
\hline
\textbf{Champ} & \textbf{Type} & \textbf{Contraintes} & \textbf{Description} \\
\hline
\_id & ObjectId & PK, Auto & Identifiant unique \\
originalName & String & Required & Nom original du fichier \\
filename & String & Unique, Required & Nom stocké (UUID) \\
mimeType & String & Required & Type MIME \\
size & Number & Required & Taille en octets \\
path & String & Required & Chemin de stockage \\
uploadedBy & ObjectId & Ref: User, Required & Utilisateur uploader \\
associatedWith.model & String & Enum: Event, User & Type d'entité liée \\
associatedWith.id & ObjectId & Required & ID de l'entité liée \\
isPublic & Boolean & Default: false & Visibilité publique \\
\hline
\end{tabular}
\caption{Structure de la collection Files}
\end{table}

\subsection{Index de Base de Données}

\begin{table}[H]
\centering
\begin{tabular}{|l|l|l|}
\hline
\textbf{Collection} & \textbf{Index} & \textbf{Type} \\
\hline
Users & email & Unique \\
Users & role & Simple \\
Users & createdAt & Descendant \\
Events & title, description & Text \\
Events & category & Simple \\
Events & status & Simple \\
Events & startDate & Simple \\
Events & organizer & Simple \\
Events & location.city & Simple \\
Registrations & (user, event) & Unique Composé \\
Registrations & event & Simple \\
Registrations & user & Simple \\
Files & uploadedBy & Simple \\
Files & (associatedWith.model, associatedWith.id) & Composé \\
\hline
\end{tabular}
\caption{Index de la base de données}
\end{table}

\newpage
% ============================================================================
\section{Diagramme de Classes}
% ============================================================================

\subsection{Architecture Backend}

\begin{figure}[H]
\centering
\resizebox{\textwidth}{!}{
\begin{tikzpicture}[
    class/.style={rectangle, draw=primary, fill=white, thick, minimum width=3.5cm, text width=3.3cm, align=center, font=\footnotesize},
    classheader/.style={rectangle, draw=primary, fill=primary!20, thick, minimum width=3.5cm, font=\bfseries\small},
    arrow/.style={-{Stealth[length=2mm]}, thick},
    node distance=0.3cm
]

% Models - Ligne 1
\node[classheader] (usermodel) {User (Model)};
\node[class, below=0 of usermodel] (userbody) {
    - email: String\\
    - password: String\\
    - firstName: String\\
    - lastName: String\\
    - role: String\\
    \hrule
    + comparePassword()\\
    + findByEmail()
};

\node[classheader, right=2.5cm of usermodel] (eventmodel) {Event (Model)};
\node[class, below=0 of eventmodel] (eventbody) {
    - title: String\\
    - description: String\\
    - category: String\\
    - status: String\\
    - startDate: Date\\
    - capacity: Number\\
    \hrule
    + isFull: Virtual\\
    + availableSpots: Virtual
};

% Services - Ligne 2
\node[classheader, below=1.5cm of userbody] (authservice) {AuthService};
\node[class, below=0 of authservice] (authbody) {
    + register(userData)\\
    + login(email, pwd)\\
    + refreshToken(token)\\
    + logout(userId)\\
    - sanitizeUser(user)
};

\node[classheader, below=1.5cm of eventbody] (eventservice) {EventService};
\node[class, below=0 of eventservice] (eventservicebody) {
    + create(data, orgId)\\
    + findAll(query)\\
    + findById(id)\\
    + update(id, data)\\
    + delete(id)
};

% Controllers - Ligne 3
\node[classheader, below=1.5cm of authbody] (authctrl) {AuthController};
\node[class, below=0 of authctrl] (authctrlbody) {
    + register(req, res)\\
    + login(req, res)\\
    + refresh(req, res)\\
    + logout(req, res)
};

\node[classheader, below=1.5cm of eventservicebody] (eventctrl) {EventController};
\node[class, below=0 of eventctrl] (eventctrlbody) {
    + getEvents(req, res)\\
    + getEvent(req, res)\\
    + createEvent(req, res)\\
    + updateEvent(req, res)
};

% Relations - flèches sur les côtés pour éviter le chevauchement
\draw[arrow] (authctrl.west) -- ++(-0.5,0) |- (authservice.west);
\draw[arrow] (eventctrl.east) -- ++(0.5,0) |- (eventservice.east);
\draw[arrow] (authservice.west) -- ++(-0.8,0) |- (usermodel.west);
\draw[arrow] (eventservice.east) -- ++(0.8,0) |- (eventmodel.east);

\end{tikzpicture}
}
\caption{Diagramme de classes simplifié - Architecture MVC}
\end{figure}

\subsection{Middlewares}

\begin{figure}[H]
\centering
\begin{tikzpicture}[
    middleware/.style={rectangle, draw=secondary, fill=secondary!10, thick, minimum width=3.5cm, minimum height=1.2cm, text width=3.3cm, align=center},
    node distance=1cm
]

\node[middleware] (auth) {\textbf{authenticate}\\Vérification JWT};
\node[middleware, right=1cm of auth] (optauth) {\textbf{optionalAuth}\\Auth optionnelle};
\node[middleware, right=1cm of optauth] (authorize) {\textbf{authorize}\\Contrôle RBAC};

\node[middleware, below=1cm of auth] (validate) {\textbf{validate}\\Validation Joi};
\node[middleware, right=1cm of validate] (upload) {\textbf{uploadSingle}\\Upload Multer};
\node[middleware, right=1cm of upload] (error) {\textbf{errorHandler}\\Gestion erreurs};

\end{tikzpicture}
\caption{Middlewares de l'application}
\end{figure}

\newpage
% ============================================================================
\section{Diagrammes de Séquence}
% ============================================================================

\subsection{Authentification - Inscription}

\begin{figure}[H]
\centering
\begin{tikzpicture}[
    actor/.style={rectangle, draw=primary, fill=primary!10, minimum width=1.2cm, minimum height=0.7cm, font=\small},
    component/.style={rectangle, draw=secondary, fill=secondary!10, minimum width=1.5cm, minimum height=0.7cm, font=\small},
    lifeline/.style={dashed, gray},
    message/.style={-{Stealth[length=2mm]}, thick},
    return/.style={-{Stealth[length=2mm]}, thick, dashed},
    note/.style={rectangle, draw=gray, fill=yellow!20, font=\small\itshape, text width=3cm, align=center}
]

% Acteurs et composants
\node[actor] (client) {Client};
\node[component, right=1.5cm of client] (controller) {Controller};
\node[component, right=1.5cm of controller] (service) {Service};
\node[component, right=1.5cm of service] (model) {Model};
\node[component, right=1.5cm of model] (db) {MongoDB};

% Lignes de vie
\draw[lifeline] (client.south) -- ++(0,-10);
\draw[lifeline] (controller.south) -- ++(0,-10);
\draw[lifeline] (service.south) -- ++(0,-10);
\draw[lifeline] (model.south) -- ++(0,-10);
\draw[lifeline] (db.south) -- ++(0,-10);

% Messages
\draw[message] ([yshift=-1.5cm]client.south) -- ([yshift=-1.5cm]controller.south) node[midway, above, font=\small] {POST /register};
\draw[message] ([yshift=-2.2cm]controller.south) -- ([yshift=-2.2cm]service.south) node[midway, above, font=\small] {register()};
\draw[message] ([yshift=-2.9cm]service.south) -- ([yshift=-2.9cm]model.south) node[midway, above, font=\small] {findOne()};
\draw[message] ([yshift=-3.6cm]model.south) -- ([yshift=-3.6cm]db.south) node[midway, above, font=\small] {query};
\draw[return] ([yshift=-4.3cm]db.south) -- ([yshift=-4.3cm]model.south) node[midway, above, font=\small] {null};
\draw[message] ([yshift=-5cm]service.south) -- ([yshift=-5cm]model.south) node[midway, above, font=\small] {create()};
\draw[message] ([yshift=-5.7cm]model.south) -- ([yshift=-5.7cm]db.south) node[midway, above, font=\small] {insert};
\draw[return] ([yshift=-6.4cm]db.south) -- ([yshift=-6.4cm]model.south) node[midway, above, font=\small] {user};
\draw[return] ([yshift=-7.1cm]model.south) -- ([yshift=-7.1cm]service.south) node[midway, above, font=\small] {user};
\draw[return] ([yshift=-7.8cm]service.south) -- ([yshift=-7.8cm]controller.south) node[midway, above, font=\small] {\{user, tokens\}};
\draw[return] ([yshift=-8.5cm]controller.south) -- ([yshift=-8.5cm]client.south) node[midway, above, font=\small] {201 Created};

\end{tikzpicture}
\caption{Diagramme de séquence - Inscription utilisateur}
\end{figure}

\subsection{Authentification - Connexion}

\begin{figure}[H]
\centering
\begin{tikzpicture}[
    actor/.style={rectangle, draw=primary, fill=primary!10, minimum width=1.2cm, minimum height=0.7cm, font=\small},
    component/.style={rectangle, draw=secondary, fill=secondary!10, minimum width=1.5cm, minimum height=0.7cm, font=\small},
    lifeline/.style={dashed, gray},
    message/.style={-{Stealth[length=2mm]}, thick},
    return/.style={-{Stealth[length=2mm]}, thick, dashed}
]

\node[actor] (client) {Client};
\node[component, right=1.5cm of client] (controller) {Controller};
\node[component, right=1.5cm of controller] (service) {Service};
\node[component, right=1.5cm of service] (model) {User};
\node[component, right=1.5cm of model] (jwt) {JWT};

\draw[lifeline] (client.south) -- ++(0,-9);
\draw[lifeline] (controller.south) -- ++(0,-9);
\draw[lifeline] (service.south) -- ++(0,-9);
\draw[lifeline] (model.south) -- ++(0,-9);
\draw[lifeline] (jwt.south) -- ++(0,-9);

\draw[message] ([yshift=-1.5cm]client.south) -- ([yshift=-1.5cm]controller.south) node[midway, above, font=\small] {POST /login};
\draw[message] ([yshift=-2.2cm]controller.south) -- ([yshift=-2.2cm]service.south) node[midway, above, font=\small] {login(email, pwd)};
\draw[message] ([yshift=-2.9cm]service.south) -- ([yshift=-2.9cm]model.south) node[midway, above, font=\small] {findByEmail()};
\draw[return] ([yshift=-3.6cm]model.south) -- ([yshift=-3.6cm]service.south) node[midway, above, font=\small] {user};
\draw[message] ([yshift=-4.3cm]service.south) -- ([yshift=-4.3cm]model.south) node[midway, above, font=\small] {comparePassword()};
\draw[return] ([yshift=-5cm]model.south) -- ([yshift=-5cm]service.south) node[midway, above, font=\small] {true};
\draw[message] ([yshift=-5.7cm]service.south) -- ([yshift=-5.7cm]jwt.south) node[midway, above, font=\small] {generateTokens()};
\draw[return] ([yshift=-6.4cm]jwt.south) -- ([yshift=-6.4cm]service.south) node[midway, above, font=\small] {tokens};
\draw[return] ([yshift=-7.1cm]service.south) -- ([yshift=-7.1cm]controller.south) node[midway, above, font=\small] {\{user, tokens\}};
\draw[return] ([yshift=-7.8cm]controller.south) -- ([yshift=-7.8cm]client.south) node[midway, above, font=\small] {200 OK};

\end{tikzpicture}
\caption{Diagramme de séquence - Connexion utilisateur}
\end{figure}

\subsection{Gestion des Événements - Création}

\begin{figure}[H]
\centering
\begin{tikzpicture}[
    actor/.style={rectangle, draw=primary, fill=primary!10, minimum width=1cm, minimum height=0.7cm, font=\small},
    component/.style={rectangle, draw=secondary, fill=secondary!10, minimum width=1.3cm, minimum height=0.7cm, font=\small},
    lifeline/.style={dashed, gray},
    message/.style={-{Stealth[length=2mm]}, thick},
    return/.style={-{Stealth[length=2mm]}, thick, dashed}
]

\node[actor] (client) {Client};
\node[component, right=1cm of client] (auth) {Auth};
\node[component, right=1cm of auth] (valid) {Valid};
\node[component, right=1cm of valid] (ctrl) {Ctrl};
\node[component, right=1cm of ctrl] (svc) {Svc};
\node[component, right=1cm of svc] (db) {MongoDB};

\draw[lifeline] (client.south) -- ++(0,-9);
\draw[lifeline] (auth.south) -- ++(0,-9);
\draw[lifeline] (valid.south) -- ++(0,-9);
\draw[lifeline] (ctrl.south) -- ++(0,-9);
\draw[lifeline] (svc.south) -- ++(0,-9);
\draw[lifeline] (db.south) -- ++(0,-9);

\draw[message] ([yshift=-1.2cm]client.south) -- ([yshift=-1.2cm]auth.south) node[midway, above, font=\scriptsize] {POST /events};
\draw[message] ([yshift=-1.9cm]auth.south) -- ([yshift=-1.9cm]valid.south) node[midway, above, font=\scriptsize] {verify JWT};
\draw[message] ([yshift=-2.6cm]valid.south) -- ([yshift=-2.6cm]ctrl.south) node[midway, above, font=\scriptsize] {validate};
\draw[message] ([yshift=-3.3cm]ctrl.south) -- ([yshift=-3.3cm]svc.south) node[midway, above, font=\scriptsize] {create()};
\draw[message] ([yshift=-4cm]svc.south) -- ([yshift=-4cm]db.south) node[midway, above, font=\scriptsize] {insert};
\draw[return] ([yshift=-4.7cm]db.south) -- ([yshift=-4.7cm]svc.south) node[midway, above, font=\scriptsize] {event};
\draw[message] ([yshift=-5.4cm]svc.south) -- ([yshift=-5.4cm]db.south) node[midway, above, font=\scriptsize] {populate};
\draw[return] ([yshift=-6.1cm]db.south) -- ([yshift=-6.1cm]svc.south) node[midway, above, font=\scriptsize] {event+org};
\draw[return] ([yshift=-6.8cm]svc.south) -- ([yshift=-6.8cm]ctrl.south) node[midway, above, font=\scriptsize] {event};
\draw[return] ([yshift=-7.5cm]ctrl.south) -- ([yshift=-7.5cm]client.south) node[midway, above, font=\scriptsize] {201 Created};

\end{tikzpicture}
\caption{Diagramme de séquence - Création d'événement}
\end{figure}

\subsection{Inscription à un Événement}

\begin{figure}[H]
\centering
\begin{tikzpicture}[
    actor/.style={rectangle, draw=primary, fill=primary!10, minimum width=1.2cm, minimum height=0.7cm, font=\small},
    component/.style={rectangle, draw=secondary, fill=secondary!10, minimum width=1.4cm, minimum height=0.7cm, font=\small},
    lifeline/.style={dashed, gray},
    message/.style={-{Stealth[length=2mm]}, thick},
    return/.style={-{Stealth[length=2mm]}, thick, dashed},
    alt/.style={rectangle, draw=gray, dashed, minimum width=8cm, minimum height=2cm}
]

\node[actor] (client) {Client};
\node[component, right=1.2cm of client] (ctrl) {Controller};
\node[component, right=1.2cm of ctrl] (svc) {Service};
\node[component, right=1.2cm of svc] (event) {Event};
\node[component, right=1.2cm of event] (reg) {Registr.};

\draw[lifeline] (client.south) -- ++(0,-10);
\draw[lifeline] (ctrl.south) -- ++(0,-10);
\draw[lifeline] (svc.south) -- ++(0,-10);
\draw[lifeline] (event.south) -- ++(0,-10);
\draw[lifeline] (reg.south) -- ++(0,-10);

\draw[message] ([yshift=-1.2cm]client.south) -- ([yshift=-1.2cm]ctrl.south) node[midway, above, font=\scriptsize] {POST /:id/register};
\draw[message] ([yshift=-1.9cm]ctrl.south) -- ([yshift=-1.9cm]svc.south) node[midway, above, font=\scriptsize] {register()};
\draw[message] ([yshift=-2.6cm]svc.south) -- ([yshift=-2.6cm]event.south) node[midway, above, font=\scriptsize] {findById()};
\draw[return] ([yshift=-3.3cm]event.south) -- ([yshift=-3.3cm]svc.south) node[midway, above, font=\scriptsize] {event};
\draw[message] ([yshift=-4cm]svc.south) -- ([yshift=-4cm]reg.south) node[midway, above, font=\scriptsize] {findOne()};
\draw[return] ([yshift=-4.7cm]reg.south) -- ([yshift=-4.7cm]svc.south) node[midway, above, font=\scriptsize] {null};
\draw[message] ([yshift=-5.4cm]svc.south) -- ([yshift=-5.4cm]reg.south) node[midway, above, font=\scriptsize] {create()};
\draw[return] ([yshift=-6.1cm]reg.south) -- ([yshift=-6.1cm]svc.south) node[midway, above, font=\scriptsize] {registration};
\draw[message] ([yshift=-6.8cm]svc.south) -- ([yshift=-6.8cm]event.south) node[midway, above, font=\scriptsize] {incrementReg()};
\draw[return] ([yshift=-7.5cm]event.south) -- ([yshift=-7.5cm]svc.south) node[midway, above, font=\scriptsize] {event};
\draw[return] ([yshift=-8.2cm]svc.south) -- ([yshift=-8.2cm]ctrl.south) node[midway, above, font=\scriptsize] {registration};
\draw[return] ([yshift=-8.9cm]ctrl.south) -- ([yshift=-8.9cm]client.south) node[midway, above, font=\scriptsize] {201 Created};

\end{tikzpicture}
\caption{Diagramme de séquence - Inscription à un événement}
\end{figure}

\newpage
% ============================================================================
\section{Sécurité}
% ============================================================================

\subsection{Mécanismes d'Authentification}

\begin{table}[H]
\centering
\begin{tabular}{|l|l|}
\hline
\textbf{Mécanisme} & \textbf{Implémentation} \\
\hline
Hashage mot de passe & bcrypt avec salt (12 rounds) \\
Token d'accès & JWT signé (HS256), expire en 15min \\
Token de rafraîchissement & JWT signé, expire en 7 jours \\
Stockage refresh token & Base de données (champ user) \\
\hline
\end{tabular}
\caption{Mécanismes d'authentification}
\end{table}

\subsection{Contrôle d'Accès (RBAC)}

\begin{table}[H]
\centering
\begin{tabular}{|l|c|c|}
\hline
\textbf{Action} & \textbf{User} & \textbf{Admin} \\
\hline
Voir événements publiés & \checkmark & \checkmark \\
Créer un événement & \checkmark & \checkmark \\
Modifier ses événements & \checkmark & \checkmark \\
Modifier tous les événements & & \checkmark \\
S'inscrire à un événement & \checkmark & \checkmark \\
Voir toutes les statistiques & & \checkmark \\
Accéder aux stats admin & & \checkmark \\
\hline
\end{tabular}
\caption{Matrice des permissions}
\end{table}

\subsection{Autres Mesures de Sécurité}

\begin{itemize}
    \item \textbf{Helmet}: Headers HTTP sécurisés
    \item \textbf{CORS}: Configuration restrictive avec origine autorisée
    \item \textbf{Rate Limiting}: 100 requêtes / 15 minutes par IP
    \item \textbf{Validation}: Joi pour toutes les entrées utilisateur
    \item \textbf{Upload}: Validation MIME type, limite de taille (5MB)
\end{itemize}

\newpage
% ============================================================================
\section{Annexes}
% ============================================================================

\subsection{Variables d'Environnement}

\begin{lstlisting}[basicstyle=\small\ttfamily, frame=single, backgroundcolor=\color{codebg}]
# Server
PORT=5000
NODE_ENV=development

# Database
MONGODB_URI=mongodb://localhost:27017/eventhub

# JWT
JWT_SECRET=your-secret-key
JWT_EXPIRES_IN=15m
JWT_REFRESH_SECRET=your-refresh-secret
JWT_REFRESH_EXPIRES_IN=7d

# Frontend
FRONTEND_URL=http://localhost:5173

# Upload
UPLOAD_PATH=uploads
MAX_FILE_SIZE=5242880
\end{lstlisting}

\subsection{Scripts NPM}

\begin{lstlisting}[basicstyle=\small\ttfamily, frame=single, backgroundcolor=\color{codebg}]
# Backend
npm run dev          # Developpement avec nodemon
npm start            # Production
npm run seed         # Peupler la base de donnees

# Frontend  
npm run dev          # Serveur de developpement Vite
npm run build        # Build de production
npm run preview      # Preview du build
\end{lstlisting}

\end{document}
